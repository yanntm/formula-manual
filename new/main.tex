\documentclass{article}

\usepackage{syntax}

%\usepackage{amssymb}
%\usepackage{amsmath}
\usepackage{dsfont}
\usepackage{xspace} 
\usepackage{xparse}
%\usepackage{subcaption}
\usepackage{times}
\usepackage{newtxtext,newtxmath}

\newcommand{\rarc}[1]{\ensuremath{\xrightarrow{#1}}\xspace} % right pointing arc with label \arc{t}
\newcommand{\ctlformulas}{\ensuremath{\Phi_{\mathit{CTL}}}\xspace} % set of CTL formulas

\newcommand{\tran}[3]{#1 \xrightarrow{\, #2 \,} #3}
\newcommand{\tuple}[1]{\langle #1 \rangle}
%\newcommand{\trans}[1]{\ensuremath{\xrightarrow{\scriptstyle {#1}}}}


\NewDocumentCommand{\rtrans}{oo} 
{\IfNoValueTF {#1}
        {\ensuremath{\xrightarrow{}}\xspace}
        {\IfNoValueTF {#2}
            {\ensuremath{\xrightarrow{#1}}\xspace}
            {\ensuremath{\xrightarrow[#2]{#1}}\xspace}}}

\NewDocumentCommand{\eval}{oo}
{\IfNoValueTF {#1}
        {\ensuremath{\mathit{eval}}\xspace}
        {\IfNoValueTF {#2}
            {\ensuremath{\mathit{eval}_{#1}}\xspace}
            {\ensuremath{\mathit{eval}_{#1}(#2)}\xspace}}}

% basic math : natural, relative integers, reals, Boolean
\newcommand{\nat}%
{\ensuremath{\mathds{N}}}
\newcommand{\zrel}%
{\ensuremath{\mathds{Z}}}
\newcommand{\rea}%
{\ensuremath{\mathds{R}}}
\newcommand{\bool}%
{\ensuremath{\mathds{B}}}

% LTS
\newcommand{\lts}{\ensuremath{TS}\xspace} % name of any LTS
\newcommand{\ltsstates}{\ensuremath{\mathcal{S}}\xspace} % set of all markings
\newcommand{\ltslabels}{\ensuremath{A}\xspace} % set of all markings
\newcommand{\ltsedges}{\ensuremath{\mathcal{\rightarrow}}\xspace} % set of all markings

\newcommand{\deadlock}{\ensuremath{\mathit{deadlock}}\xspace}
\newcommand{\true}{\ensuremath{\mathit{true}}\xspace}
\newcommand{\false}{\ensuremath{\mathit{false}}\xspace}
\newcommand{\EF}{\ensuremath{\mathit{EF}}\xspace}
\newcommand{\AG}{\ensuremath{\mathit{AG}}\xspace}
\newcommand{\A}{\ensuremath{\mathit{A}}\xspace}

\newcommand{\pathlen}{\ensuremath{\ell}\xspace}
\newcommand{\paths}{\ensuremath{\Pi}\xspace}
\newcommand{\maxpaths}{\ensuremath{\Pi^{\mathit{max}}}\xspace}
\newcommand{\en}{\ensuremath{en}\xspace} 
\newcommand{\props}{\ensuremath{\mathit{AP}}\xspace}
\newcommand{\apeval}{\ensuremath{\mathit{v}}\xspace}


% macro pour Petri
\newcommand{\Places}{\ensuremath{\mathcal{P}}}
\newcommand{\Trans}{\ensuremath{\mathcal{T}}}
\newcommand{\Pre}{\ensuremath{\mathcal{W}_-}}
\newcommand{\Post}{\ensuremath{\mathcal{W}_+}}
\newcommand{\Effect}{\ensuremath{\mathcal{W}_e}}
\newcommand{\tPre}{\ensuremath{\mathcal{W}_-^\top}}
\newcommand{\tPost}{\ensuremath{\mathcal{W}_+^\top}}
\newcommand{\Preset}[1]{\ensuremath{\bullet{#1}}}
\newcommand{\Postset}[1]{\ensuremath{{#1}\bullet}}

% reachable states
\newcommand{\Reach}{\ensuremath{\mathcal{R}}}
\newcommand{\Stutter}{\ensuremath{\mathcal{S}_{t}}}

% logic symbols
\renewcommand{\implies}{\Rightarrow}
\renewcommand{\impliedby}{\Leftarrow}
\renewcommand{\iff}{\Leftrightarrow}



\title{Property Semantics for the Use in MCC}

\begin{document}
\maketitle


\begin{abstract}
This document presents the official semantics for properties used in the Model-Checking Contest (MCC).

\end{abstract}

\section{Introduction}


\section{Syntax}

\subsection{Place Transition Nets}

\subsection{Colored Nets}

\subsection{Syntactic Expansion of Colored Nets to Place Transition Nets}

\section{Semantics}

\subsection{Reachable Marking Graph of a PT net}

\subsection{Atomic Propositions on a marking}

Finally, we fix the set of atomic propositions $\alpha$ ($\alpha \in \props$) 
for Petri nets as used in the MCC Property Language:
\begin{grammar}
    \let\syntleft\relax
    \let\syntright\relax
    <$\alpha$> ::= \textit{t} |  $e_1 \bowtie e_2$

    <$e$> ::= $c$ | $p$ | $e_1 \oplus e_2$
\end{grammar}
where  $t$ is a transition, $c \in \mathbb{N}^0$, $\bowtie$ $\in$ $\{<, \leq, =, \neq, >, \geq\}$, $p$ is a place, and $\oplus \in \{ +, -, * \}$.
The evaluation function $\apeval$ for a marking $M$ is given as
$\apeval(M) = \{ t \in T \mid t \in \en(M) \} \cup
\{ e_1 \bowtie e_2 \mid \eval[M][e_1] \bowtie \eval[M][e_2]\}$
where $\eval[M][c] = c$, $\eval[M][p] = M(p)$ and
$\eval[M][e_{1} \oplus e_{2}] = \eval[M][e_{1}] \oplus \eval[M][e_{2}]$.


\subsection{Labelled Kripke Structure}

A \emph{labelled transition system} (LTS) is a 
tuple $(\ltsstates, \ltslabels, \ltsedges)$ where $\ltsstates$ is a 
set of states, $\ltslabels$ is a set of actions (or labels), 
and $\ltsedges{} \subseteq \ltsstates \times \ltslabels \times \ltsstates$ is a transition relation. We write $s \rtrans[a] s'$ 
whenever $(s,a,s') \in \ltsedges$ and say that $a$ is \emph{enabled} in $s$.
The set of all enabled actions in a state $s$ is denoted $\en(s)$.
We write $s \rtrans s'$ whenever there is an action $a$ such that
$s \rtrans[a] s'$.

A \emph{run} starting at $s_0$ is any finite or infinite sequence
$s_0 \rtrans[a_0] s_1 \rtrans[a_1] s_2 \rtrans[a_2] \cdots$ where
$s_0, s_1, s_2, \ldots \in \ltsstates$, 
$a_0, a_1, a_2 \cdots \in \ltslabels$ and
$(s_i,a_i,s_{i+1}) \in \ltsedges$ for all respective $i$.
We use $\paths(s)$ to denote the set of all runs starting at the state $s$.
A run is \emph{maximal} if it is either infinite or ends in a state that
is a deadlock.  Let $\maxpaths(s)$ denote the set of all maximal 
runs starting at the state $s$.
A \emph{position} $i$ in a run $\pi = 
s_0 \rtrans[a_0] s_1 \rtrans[a_1] s_2 \rtrans[a_2] \cdots$
refers to the state $s_i$ in the path and is written as $\pi_i$.
If $\pi$ is infinite then any $i$, $0 \leq i$, is a position in $\pi$.
Otherwise $0 \leq i \leq n$ where $s_n$ is the last state in $\pi$.

\section{CTL}

We now define the syntax and semantics of a \emph{computation tree logic} 
as used in the Model Checking Contest.
Let \props be a set of atomic propositions.
We evaluate atomic propositions on a given LTS
$\lts = (\ltsstates, \ltslabels, \ltsedges)$ 
by the function $\apeval : \ltsstates \rarc{} 2^{\props}$ so that
 $\apeval(s)$ is the set of atomic propositions satisfied 
in the state $s \in \ltsstates$.

The CTL syntax is given as follows (where $\alpha \in \props$ ranges over atomic propositions):
\newcommand{\indalt}[1][2]{\\\hspace*{1pt}\textbar\hspace*{3.5pt}}
%\setlength{\grammarindent}{3em} % increase separation between LHS/RHS 
\begin{grammar}
    \let\syntleft\relax
    \let\syntright\relax

    <$\varphi$> ::= \true | \false | $\alpha$ |$\deadlock$ | $\varphi_{1} \land \varphi_{2}$ | $\varphi_{1} \lor \varphi_{2}$ | $ \neg\varphi$ 
%| $\varphi_{1} \mathord{\implies} \varphi_{2}$ | $\varphi_1 \mathord{\iff} \varphi_2$  
| $\mathit{AX}\varphi$ | $\mathit{EX}\varphi$ | $\mathit{AF}\varphi$ | $\mathit{EF}\varphi$ | $\mathit{AG}\varphi$ | $\mathit{EG}\varphi$ | $\mathit{A}(\varphi_1\mathit{U}\varphi_2)$ | $\mathit{E}(\varphi_1\mathit{U}\varphi_2)$ \ .
\end{grammar}
We use \ctlformulas to denote the set of all CTL formulae.
The semantics of a CTL formula $\varphi$ in a state $s \in \ltsstates$ 
is given in Table~\ref{tab:CTLsemantics}. 

\begin{table}[htbp]
\raggedbottom
\begin{align*} 
    &s \models \true && \\
    &s \not\models \false &&  \\
%    &s \models a && \mbox{iff } a \in \en(s) \\
    &s \models \alpha && \mbox{iff } \alpha \in \apeval(s) \\
	&s \models \deadlock &&\mbox{iff } \en(s) = \emptyset \\
	&s \models \varphi_1 \land \varphi_2 &&\mbox{iff } s \models \varphi_1 \mbox{ and } s \models \varphi_2 \\
	&s \models \varphi_1 \lor \varphi_2 &&\mbox{iff } s \models \varphi_1 \mbox{ or } s \models \varphi_2 \\
	&s \models \neg \varphi&&\mbox{iff } s \not\models \varphi\\
% &s \models \varphi_1 \mathord{\implies} \varphi_2 &&\mbox{iff } s \not\models \varphi_1 \mbox{ or } s \models \varphi_2 \\
% &s \models \varphi_1 \mathord{\iff} \varphi_2 &&\mbox{iff } (s \models \varphi_1 \mbox{ iff } s \models \varphi_2) \\
	&s \models \mathit{AX}\varphi&& \mbox{iff } s' \models \varphi \mbox{ for all } s' \in \ltsstates \mbox{ s.t. } s \rtrans s' \\ 
&s \models \mathit{EX}\varphi&&\mbox{iff there is } s' \in \ltsstates \mbox{ s.t } s \rtrans s' \mbox{ and } s' \models \varphi \\
	&s \models \mathit{AF}\varphi&&\mbox{iff for all }\pi \in \maxpaths(s) \mbox{ there is a position } i \mbox{ in } \pi \mbox{ s.t. } \pi_i \models \varphi \\
	&s \models \mathit{EF}\varphi&&\mbox{iff there is } \pi \in \maxpaths(s) \mbox{ and a position } i \mbox{ in } \pi \mbox{ s.t. } \pi_i \models \varphi \\
	&s \models \mathit{AG}\varphi&&\mbox{iff for all } \pi \in \maxpaths(s) \mbox{ and for all positions } i \mbox{ in } \pi \mbox{ we have } \pi_i \models \varphi \\
	&s \models \mathit{EG}\varphi &&\mbox{iff there is }\pi \in \maxpaths(s) \mbox{ s.t. for all positions } i \mbox{ in } \pi \mbox{ we have } \pi_i \models \varphi \\
	&s \models \mathit{A}(\varphi_1 \mathit{U} \varphi_2) &&\mbox{iff for all }\pi \in \maxpaths(s) \mbox{ there is a position } i \mbox{  in } \pi  \mbox{ s.t. }\\
	& && \pi_i \models \varphi_2 \mbox{ and for all $j$, } 0 \leq j < i, \mbox{ we have } \pi_j \models \varphi_1 \\
	&s \models \mathit{E}(\varphi_1 \mathit{U} \varphi_2) &&\mbox{iff there is }\pi \in \maxpaths(s) \mbox{ and there is a position } i \mbox{ in } \pi \mbox{ s.t. } \\
	& && \pi_i \models \varphi_2 \mbox{ and for all $j$, } 0 \leq j < i, \mbox{ we have } \pi_j \models \varphi_1
\end{align*}
\caption{Semantics of CTL formulae} \label{tab:CTLsemantics}
\end{table}

%Let $N = \petriinhibtuple$ be a Petri net. 
\paragraph{Acknowledgement.} Definitions for CTL are taken from~\cite{BDJJS:PN:18}.

\section{LTL}



\textbf{Definition}
\emph{Trace}
A trace is an infinite sequence of markings connected by the one-step
firing relation. The sequence may contain a deadlock marking which is
then repeated forever.

Formally : Let $\pi = (m_1 m_2 m_3 \ldots)$ be an infinite sequence of markings.
It is a trace if $\forall i \in \nat,  i \geq 1$, $m_{i} \rtrans m_{i+1}$ or $m_i$ is a \emph{deadlock} state and $m_{i} = m_{i+1}$.
\medskip


\textbf{Definition}
\emph{Suffix of a trace}

A suffix of a trace is obtained from a given trace by removing the first n
markings of the sequence, for some natural number n.

Formally : Let $\pi = (m_1 m_2 m_3 \ldots)$ be a trace and $n \in \nat$, trace $\pi+n$
is defined as $(m_{n+1} m_{n+2} m_{n+3} \ldots)$.

For all natural $n$, $\pi+n$ is called the \emph{suffix of $\pi$}. 
\medskip

We consider that $\mathit{F}\varphi$ is interpreted semantically as $\true \mathit{U} \varphi$.

The semantics of a LTL formula $\varphi$ in a trace $\pi = (m_1 m_2 m_3 \ldots)$ 
is given in Table~\ref{tab:LTLsemantics}. 

\begin{table}[htbp]
\raggedbottom
\begin{align*} 
    &\pi \models \true && \\
    &\pi \not\models \false &&  \\
    &\pi \models \alpha && \mbox{iff }& \alpha \in \apeval(m_1) \\
	&\pi \models \varphi_1 \land \varphi_2 &&\mbox{iff } &\pi \models \varphi_1 \mbox{ and } \pi \models \varphi_2 \\
	&\pi \models \varphi_1 \lor \varphi_2 &&\mbox{iff }& \pi \models \varphi_1 \mbox{ or } \pi \models \varphi_2 \\
	&\pi \models \neg \varphi&&\mbox{iff }& \mbox{ not } \pi \models \varphi\\
	&\pi \models \mathit{X}\varphi&& \mbox{iff }& \pi+1 \models \varphi  \\ 
	&\pi \models \mathit{G}\varphi&& \mbox{iff }& \forall i \in \nat, \pi+i \models \varphi  \\ 
	&\pi \models \varphi_1 \mathit{U}\varphi_2&& \mbox{iff } &\exists i \in \nat,  \pi+i \models \varphi_2 \mbox{ and } 
		 \forall j \in \nat, j < i, \pi+j \models \varphi_1   
\end{align*}
\caption{Semantics of LTL formulae} \label{tab:LTLsemantics}
\end{table}


Let $m_0$ be the initial marking of $N$.
Trace $\pi = (m_1 m_2 m_3 \ldots)$ is called initial trace if $m0 = m1$.
For an LTL formula $\varphi$, $N \models \varphi$ (equivalently: $N \models \A \varphi$ , where $\A$ is the
universal path quantifier) if, for all initial traces $\pi, \pi \models \varphi$.



Our definitions implicitly transform the reachability graph of N into a
Kripke structure. A Kripke structure is a transition system where every
state has at least one successor state. The transformation consists of
adding, for each deadlock marking m, m as its own successor (self-
loop). We have implemented this transformation in our definition of
"trace". For Kripke structures, the definition of LTL semantics is
undisputed, to our best knowledge.

\paragraph{Acknowledgement.} Definitions for LTL are adapted from Karsten Wolf..


\bibliographystyle{plain}
\bibliography{references.bib}


\end{document}
